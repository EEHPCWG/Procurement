\begin{itemize}

\item[(info)]
The vendor shall provide [equipment, services and/or resources] that – among other objectives – establish a highly energy efficient solution at justifiable cost.  The proposed solutions should demonstrate net benefits under normal production conditions.  
\end{itemize}

\section{Energy-related Total Cost of Ownership (TCO)}

\begin{itemize}

\item[(enhancing)]
It is an objective of [Customer] to encourage innovative programs whereby the vendor and/or [Customer] are incentivized to reduce the costs for energy and/or power related capital expenditures as well as the operational costs for energy.  This may be for the system, data center and/or broader site.  By doing this, the vendor would be reducing the energy-related TCO for [Customer].  The vendor is encouraged to describe their support for these innovative programs in qualitative as well as quantitative terms.

\item[(info)]
An example of an innovative program for bringing the energy/power element of TCO to the front was used by LRZ.  Their procurement was based on TCO whereby the budget covered not just investment and maintenance, but operational costs as well. The intent was to provide a clear incentive for the vendor to deliver a solution that would yield low operational costs and, thereby, lower TCO.

\end{itemize}

\section{Power Usage Effectiveness (PUE)}

\begin{itemize}

\item[(info)]
It is an objective of [Customer] to run a highly energy efficient data center.  One measure for data center efficiency is PUE.  It is recognized that the metric PUE has limitations.  For example, solutions with cooling subsystems that are built into the computing systems will result in a more favorable PUE than those that rely on external cooling, but are not necessarily more energy efficient.  In spite of these limitations, PUE is a widely adopted metric that has helped to drive energy efficiency.

\item[(enhancing)]
The US Department of Energy Office of the Chief Information Officer has set a requirement to achieve an average PUE of 1.4 by 2015.  As a result, the vendor is encouraged to qualitatively describe their support for helping [Customer] to meet this requirement.

\end{itemize}

\section{Total Usage Effectiveness (TUE)}

\begin{itemize}

\item[(info)]
TUE is another metric that has been developed to overcome the limitations of the PUE metric.  Specifically, it resolves the issue of PUE differences due to infrastructure loads moving from inside to outside the box.  TUE is the total energy into the data center divided by the total energy to the computational components inside the IT equipment.

\item[(enhancing)]
The vendor is encouraged to qualitatively describe their support for measuring TUE.
\end{itemize}

\section{Energy Re-Use Effectiveness (ERE)}

\begin{itemize}

\item[(info)]
Some sites have the ability to utilize the heat generated by the data center for productive uses, such as heating office space.  Energy re-use is not strictly adding to the energy efficiency of either the computing system or the data center, but it can reduce the energy requirements for the surrounding environment.  For those sites, it would be an objective of [Customer] to achieve an ERE < 1.0.  

\item[(enhancing)]
The vendor is encouraged to qualitatively describe their support for helping [Customer] to achieve an ERE < 1.0.  

\end{itemize}

\section{Power Distribution}
\begin{itemize}
\item[(important)]
The vendor is encouraged to qualitatively describe energy efficient and innovative solutions that help to reduce conversion losses in the data center.
\end{itemize}

