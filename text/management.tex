\begin{itemize}
\item[\textbf{(info)}]
As with the measurement capabilities described above, power and energy management and control 
capabilities (hardware and software tools and application programming interfaces (APIs)) are 
necessary to meet the needs of future supercomputing energy and power constraints. It is 
extremely important that [Customer] utilize early capabilities in this area and start 
defining and developing advanced capabilities and integrating them into a user friendly, 
production environment.  

The vendor shall provide mechanisms to manage and control the power and energy 
consumption of the system. These mechanisms may differ in implementation and purpose.  
Below are envisioned usage models for these management capabilities.  They are categorized 
loosely by where the management occurs. It is envisioned that this capability will evolve 
over time from initial monitoring and reporting capabilities, to management (including 
activities like 6-sigma continuous improvement), and even to autonomic controls.   

These usage models are not requirements for the vendor, but rather suggestive 
examples that serve to help clarify the requirements for measurement capabilities 
described in Section~\ref{sec:measurements} above. Furthermore, it is recognized that 
many of these solutions would be provided by a third party, not by the system vendor.  
\end{itemize}

\section{Datacenter/Infrastructure}
\begin{itemize}
\item[\textbf{(info)}]
Respond to utility requests or rate structures. For example, cut back usage 
during high load times, limit power during expensive utility rate times of the day. 

 ``Power capping'' the system allows for provisioning the infrastructure for closer 
to average usage, leading to substantial infrastructure savings compared to those 
centers which are designed for theoretical peak usage.  

Respond to demand requests, including increases in load to accommodate waste 
heat recovery, renewable energy, etc.

Manage rate of power changes; e.g., avoid spikes.  As another example, the large variations of 
harmonic current produced by computer loads may need to be balanced in the datacenter as well 
as the site's broader infrastructure and even the grid.  
\end{itemize}

\section{System Hardware and Software}
\begin{itemize}
\item[\textbf{(info)}]
Reduce power utilization during ``design days'' so as to enable use of free cooling 
without backup chillers.  Implement alarm and/or automatic shut-down that responds to environmental 
temperature excursions that are outside of the facility design envelope by reducing system loads.  

Identify higher than normal power draw components needing maintenance and/or replacement.   
posibly also identify higher than normal power draw usage from SW- perhaps 
that is ``stuck'' in an infinite loop-back mode.     

Proliferate power scaling and management beyond computation, to memory, communication, 
I/O and storage.  For example, consider under and overclocking and OS/hardware control of the 
total amount of energy consumed 

In addition to traditionally compiling for performance, the compiler vendor may want 
to provide the user with mechanisms to compile for energy efficiency. The possible mechanisms 
may include the following.

\begin{itemize}
\item
Compiler flags for specifying performance-energy trade-offs or regarding energy efficiency 
as an optimization goal or a constraint.
\item
Programming directives for conveying user-level information to the compiler for 
better optimization in the context of energy efficiency.
\item
Program constructs to promote energy as the first-class object so that it can be 
manipulated directly in source code.
\item
Compiler-based tools for reporting analyzed results regarding the energy efficiency of applications.     
\end{itemize}
\end{itemize}

\section{Applications, Algorithms, Libraries}
\begin{itemize}
\item[\textbf{(info)}]
Provides programming environment support that leads to enhanced energy efficiency.

Reduce wait-states. Examples are the following:

\begin{itemize}
\item
Schedule background I/O activity more efficiently with I/O interface extensions 
to mark computation and communication dominant phases. 
\item
Use an energy-aware MPI library which is able to use information of wait-states 
to reduce energy consumption.
\end{itemize}

Reduce the power draw in wait-states. An example is the following:

\begin{itemize}
\item
Attain energy reduction for task-parallel execution of dense and sparse linear algebra 
operations on multi-core and many-core processors, when idle periods are leveraged 
by promoting CPU cores to a power-saving C-state.
\end{itemize}

Scale resources appropriately. Examples are the following:
\begin{itemize}
\item
Apply the phase detection procedure to parallel electronic structure calculations, 
performed by a widely used package GAMESS. Distinguishing computation and communication 
processes have led to several insights into the role of process-core mapping in the 
application of dynamic frequency scaling during communications.
\item
Analyze the energy-saving potential by reducing the voltage and frequency of processes 
not lying on a critical path, i.e., those with wait-states before global synchronization points.
\item
Enabling network bandwidth tuning for performance and energy efficiency.
\end{itemize}

Select appropriate energy-performance trade-off. An example is the following:
\begin{itemize}
\item
Optimize the power profile of a dense linear algebra algorithm (PLASMA) by 
focusing on the specific energy requirements of the various factorization 
algorithms and their stages.
\end{itemize}

Programming and performance analysis tools. An example is the following.
\begin{itemize}
\item
Counters, accumulators, in-band support.
\end{itemize}

Open up control of these policies so that we can turn them on and off including   
zero setting if a policy is detrimental to our applications at scale. 
\end{itemize}

\section{Schedulers, Middleware, Management}
\begin{itemize}
\item[\textbf{(info)}]
Putting hardware into the lowest reasonable power state or switching off idle 
resources (nodes, storage, etc.) when job scheduling cannot allow for full utilization. 

Different power states. Careful about how we switch a power state off.  Cannot affect reliability.  
Sleep states are probably the best direction.  Response time is much better.

Energy-aware scheduling. Develop mechanism to automatically select processor 
frequency for which energy to solution is minimized for a specific application.

Demand response (as in the ability to react to electrical grid based incentives)
requires enhanced scheduling tools. 

Evolving hardware features will likely require enhanced system software and scheduling 
tools with control at all levels of the hierarchy, from the system down to the components.  
An example might be a scenario where you have a high priority job and there are available 
nodes to run the job; but if the job runs at the desired P-state, the system would exceed some 
notion of a power cap.  In this situation, can one dynamically alter the P-state of lower 
priority jobs to allow them to continue, perhaps at a slower rate, while also accommodating 
the new, high priority job.
\end{itemize}

