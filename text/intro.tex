This document captures some best practices to consider when writing procurement documents for supercomputer acquisitions. These best practices concern energy efficiency, especially capabilities to measure and manage both power and energy consumption. The document draws upon recent procurement documents created and used by two major supercomputing sites. In addition the document modifies and supplements the material from these procurement documents with input from experts in energy efficient HPC.
 
The team that wrote this draft consists exclusively of members from the user community, mostly from US DOE Labs.  General publiction will include review and feedback from vendors.

Although  progress has been made, there remains much room for improvement in HPC energy efficiency. This document sets this year’s vision (2013) for systems to be delivered and accepted in two years (2015). It identifies priorities and sets an immediate goal.  Becasue it is expected that these priorities will change and that the bar will rise over time, this document will be refreshed on a yearly basis.

Some of the content below is informational and as such is intended to set the context for the acquisition, but not to be used as a requirement.  Additional content reflects requirements and is intended to specify system features and capabilities.  These requirements are categorized as mandatory, important, or enhancing.

The intent is that this document encourage dialogue about priorities and requirements for HPC system energy efficient features and capabilities while recognizing that each HPC center has its own unique mission with differing priorities. The requirements discussed here are intended to draw lines in the sand that can be easily re-drawn, not to build isolating fences.
 
Finally, this document is intended to be high level while remaining vendor and technology neutral.  It should encourage innovation and not pick a particular vendor or solution.

The content is organized into five categories, all focused on energy efficiency.  

Measurement, benchmarks and management focus almost exclusively on system hardware and system software, but also span applications.  

The other two categories are general objectives and cooling.  These two areas span infrastructure and the supercomputer system itself (mostly system hardware, but some aspects of system software as well).

Conventions:

Information: info

\begin{packed_enum}
\item
enhancing
\item
important
\item
mandatory
\end{packed_enum}