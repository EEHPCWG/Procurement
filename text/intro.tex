This document captures ‘best practices’ for including energy efficiency and especially capabilities to measure and manage both power and energy consumption as s an important considerations when writing procurement documents for supercomputer acquisitions.  It draws upon recent procurement documents created and used by two major supercomputing sites, but also draws upon content experts in energy efficient HPC to modify and supplement the material from these documents.  The team that wrote this draft has been comprised exclusively of members in the user community, mostly from US DOE Labs.  The plan is to include vendor review and feedback prior to general publication of these best practices.

The energy efficiency of HPC systems has been improving, but the road ahead still requires improvement.  This document sets this year’s vision (2013) for systems to be delivered and accepted in two years (2015).  It identifies priorities and sets an immediate bar.  It is expected that the priorities will change and the bar will rise over time.  It is also expected that this document will be refreshed on a yearly basis.

Some of the content below is informational and would be used to set the context for the acquisition, but not be used as a requirement.  The rest of the content below reflects requirements and would be used to specify system features and capabilities.  These requirements are categorized as mandatory, important, or enhancing. 

That said, it is intended that this document encourage dialogue in the entire community about priorities and specific requirements for HPC system energy efficient features and capabilities. Each HPC center has its own unique mission, and priorities may differ greatly between users. The requirements are intended to draw lines in the sand that can be easily re-drawn, not to build isolating fences. 

Another caveat, this document is intended to be vendor and technology neutral.  It is intended to be high level.  It should encourage innovation and not pick a particular vendor or solution.  

The content is organized into five categories, all focused on energy efficiency.  Measurement, benchmarks and management focus almost exclusively on system hardware and system software, but also span applications.  The other two categories are general objectives and cooling.  These two areas span infrastructure and the supercomputer system (mostly system hardware, but some aspects of system software as well). 

Conventions:

Information: info

\begin{packed_enum}
\item
enhancing
\item
important
\item
mandatory
\end{packed_enum}

